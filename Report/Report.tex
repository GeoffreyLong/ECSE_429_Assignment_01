\documentclass[11pt, a4paper]{article}
\usepackage{times}

%Clearly state the course, term, group number, and team members on the title page of your report
\title{ECSE 429 (Software Validation) \\ Assignment One Report}
\author{Geoffrey Long (\textit{260403840}) \\ Kelly Morrison (\textit{260459446})}

%TODO rename as ConformanceTestReport.pdf and put at root level of project when completed... delete the .tex and associated files at this point
\begin{document}
\maketitle

%%%%%%% SECTIONS %%%%%%%
% Describe how to run your source code to generate the test class for a given state machine (xml file) and corresponding implementation of the state machine. This description should work for the CCoinBox example but also for the unknown state machine and its implementation.
% In a few paragraphs, discuss which manual changes you had to make to the generated test class to get the complete test class and why you had to make those manual changes instead of automatically generating the test code.
% In one paragraph, describe whether you found any defects in the implementation of the CCoinBox example. For each found defect, describe how you fixed it.
% In one paragraph, discuss what would be the main challenge to automate the generation of sneak path test cases from a given state machine conforming to the metamodel in Figure 1.


%%%%%%% Marking Scheme %%%%%%%
% How to run the source code -> 2/20
% Discussion of manual changes -> 8/20
% Discussion of any defects in CCoinBox -> 6/20
% Discussion of generation of sneak path test cases -> 4/20



\section{Running the Source Code}
Import the ConformanceTest.zip file into Eclipse using the Import feature of Eclipse (File $\rightarrow$ Import $\rightarrow$ General $\rightarrow$ Existing Projects into Workspace $\rightarrow$ Select archive file). To generate tests for a given state machine (xml format), add the full path of the xml file to the argument of the project (Run $\rightarrow$ Run Configurations $\rightarrow$ Arguments $\rightarrow$ Program Arguments), then click "apply" then click "run". This will run the main class, ConformanceTest, which automatically generates the JUnit Test file, which is called Generate$<ClassName>$ where ClassName is the name of the class associated with the xml file. You will find this Test class saved in the same package as the implementation of the state machine. Open this class and run it (Run $\rightarrow$ Run As $\rightarrow$ JUnit Test). 

\section{Discussion of Manual Changes}
When we run our automatically generated test class on the CCoinBox state machine, we have 2/10 tests that fail. This is because certain conditions are not met. Specifically, when you are in the “allowed” state of the machine, there are three transitions that execute the “vend” event. These three transitions all depend on the value of curQtrs and should execute only when the condition allows. Although a conformance test is generated for each of these paths, the round trip path up to the "allowed" state, and therefore the value of curQtrs, will be the same for all three tests. In this case, the value of curQtrs is 2 for all three transitions. The tests checking the other paths - one that requires curQtrs==3 and another for curQtrs>3 - will always fail. 

This issue was dealt with in the manual implementation of the test class, TestCCoinBox, via manipulation of the condition variable. In this class, we added a while loop to increment the value of curQtrs until its value satisfies the condition for the transition being tested. We also have an if statement that verifies that the value is correct, to ensure the assertTrue statement will pass. After adding these manual changes, our test class passes for all ten conformance tests. %TODO do we want that if statement?

%This shows that our automatically generated test class will not be able to handle cases where there are multiple duplicate transitions from a given state that vary only on condition.
This shows that our automatically generated test class will not be able to handle cases where there are multiple conditions for a given event at a specific state. Based on the previous transitions in the round trip path, the condition variable will have a specific value and will only meet certain conditions for the same event. Although it is possible to automatically generate code to "force" the values to match the conditions, this might not be desirable. It is difficult to say if a condition is failing because it is supposed to fail, or because it simply didn't take the right path through the tree. Essentially, it would be hard to automatically generate a test class that takes care of these cases without knowing the nature of the state machine. 

It is worth noting that similar behavior can be seen in the automatically generated test class of the Legislative state machine. There are 2/6 tests that fail because the value of the variable isCommonBill does not match the expected value.

\section{Discussion of CCoinBox Defects}
In the addQtr() method of the CCoinBox.java class, there is an error in the “case allowed” section (line 142). According to the diagram on the PDF and the XML file, when you are in the “allowed” state, and you call the addQtr() method, the new state should be “allowed”, but in the java code, the new state is set to “notAllowed”. We updated the value of the new state to “allowed” to match the state machine from the XML file. 

Similarly, in the reset() method and the returnQtrs() method of the CCoinBox.java class, there are errors in the “allowed” case. For both methods, when you are in the “allowed” state and the new state is “empty”, the allowVend Boolean should be set to false. This is again according to the XML file and the PDF. We added this statement to both the returnsQtrs() and reset() methods in the “allowed” cases (lines 113 and 175 respectively). 

\section{Generating Sneak Path Test Cases}
The main challenge of automatically generating the sneak path test cases from a given state machine conforming to the metamodel given to us is that we do not know the nature of the state model so it would be hard to determine which events are illegal for a given state. 

\end{document}